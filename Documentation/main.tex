\documentclass[a4paper,12pt]{article}
\usepackage[catalan]{varioref}
\usepackage{setspace}
\input{sections/packages}
%\title{
	\begin{center}
	\vspace{3cm}
	\includegraphics[width=11cm, height=3cm]{images/Logo-nou-eps.jpg}
	\end{center}
	\begin{center}
	\line(1,0){340}
	\end{center}		
	TREBALL DE FI DE GRAU\\
	\vspace{2mm}
	\Large L'alzheimer contra la IA. Qui guanyarà?\\
	\line(1,0){340}
	\vspace{2.5cm}
	}

\author{Marc Cervera Rosell - 47980320C \vspace{1cm}}


\date{Curs acadèmic 2022-23\vspace{0.5cm} \\Grau en Enginyeria Informàtica}
\onehalfspacing

\begin{document}
\thispagestyle{empty}
\includepdf{Portada-TFG}
\thispagestyle{empty}
\tableofcontents
\thispagestyle{empty}
\listoffigures
\listoftables
\thispagestyle{empty}

\newpage
\setcounter{page}{1}
\pagestyle{plain}
\section*{Introducció}
\addcontentsline{toc}{section}{Introducció}
Detectar de manera precoç l'Alzheimer en una persona és un dels desafiaments que enfronta la medicina del segle XXI. Els mètodes utilitzats avui dia per a la detecció precoç d'aquesta malaltia són molt costosos i requereixen una copiosa inversió de temps i recursos. Tot i que la detecció d'aquesta afecció és un desafiament realment complicat, és crucial per poder garantir un tractament i una qualitat de vida adequats per als pacients que, per desgràcia, pateixen els efectes d'aquesta malura.\\
Durant la realització d'aquest treball de final de grau, s'investigarà de quina manera pot ajudar la intel·ligència artificial en la detecció primerenca de la malaltia d'Alzheimer a partir d'imatges de ressonància magnètica cerebral (MRI en anglès). L'aplicació del \textit{Machine Learning} en el camp de la salut, i més concretament en el cas que tracta aquest treball (l'Alzheimer), és una tècnica que augura moltes esperances de facilitar la detecció precoç d'aquesta terrible malaltia. Les esperances es basen en el fet que l'ús de la intel·ligència artificial pot ajudar, com s'ha esmentat, a detectar de manera primerenca aquesta malaltia d'una manera no invasiva ni dolorosa per al pacient, cosa que podria millorar molt significativament la qualitat de vida i reduir els costos que suposa el diagnòstic tardà i el tractament.\\
Com s'explicarà més endavant en aquest document, les xarxes neuronals convolucionals, s'han convertit en la tècnica que proporciona més esperances a les comunitats científica i sanitària, perquè són capaces d'extreure les característiques complexes d'una ressonància magnètica i utilitzar-les per ajudar al professional mèdic corresponent a confirmar o descartar el diagnòstic d'Alzheimer.\\
Per dur a terme l'objectiu principal d'aquest treball de final de grau, es farà ús de la tècnica d'aprenentatge supervisat, per entrenar una sèrie de dades etiquetades que “ensenyaran” a la xarxa convolucional a diferenciar entre quatre nivells possibles de demència per Alzheimer. Aquests quatre nivells són; no dement, dement molt lleu, dement lleu, dement moderat.\\
L'estructura que seguirà el treball consta de vuit blocs, el primer dels quals contindrà informació sobre la malaltia d'Alzheimer. El segon, conformarà el bloc del marc teòric computacional, en el qual constarà informació detallada sobre la tecnologia emprada en la realització del treball. El tercer bloc tractarà sobre les aplicacions de la intel·ligència artificial en el camp de la salut. Concretament, es comentaran diversos aspectes relacionats amb l'aplicació de la tecnologia per al diagnòstic i tractament de la malaltia. El següent bloc contindrà informació sobre el desenvolupament de la proposta, tot mostrant anàlisis gràfiques de les mètriques de rendiment de la xarxa neuronal com una anàlisi exhaustiva d'en quines parts de les imatges es fixa el model en el moment de fer la classificació de la imatge d'input. Seguidament, el cinquè bloc contindrà les conclusions extretes després de realitzar la investigació. El sisè bloc contindrà els annexos del treball on es definiran alguns conceptes que no hagin pogut quedar clars en el moment de la redacció d'aquest document. En el penúltim bloc, constaran els agraïments a les persones que han fet possible la realització i la finalització exitosa d'aquest treball. Finalment, el vuitè i últim bloc, especificarà les referències bibliogràfiques que han estat necessàries per a la recerca.
\section*{\textit{Abstract}}
\addcontentsline{toc}{section}{Abstract}
\textit{This document will explain the development process of an artificial intelligence model able to identify up to four levels of dementia caused by Alzheimer's disease. These levels are; “non-demented”, “ver mild demented”, “mild demented” and “moderate demented”.\\
The main aim of this thesis is to give to the health workers a new tool to get helped to confirm or discart the Alzheimer's diagnosis, but the most important is not to give a diagnosis tool. The most essential, is to give a new tool that allows early diagnosis to give to the patients the best life quality and reduce the costs of late diagnosis.}
\section*{Marc teòric clínic}
\addcontentsline{toc}{section}{Marc teòric clínic}
\subsection*{Quan i qui va descobrir l'Alzheimer?}
\addcontentsline{toc}{subsection}{Quan i qui va descobrir l'Alzheimer?}
Abans d'entrar de ple en què és l'Alzheimer, quins símptomes presenta, etc. cal fer una mica la vista enrere a la història per saber qui i quan va descobrir aquesta malaltia.\\
A principis del segle XX, concretament l'any 1901, el psiquiatre alemany \textit{Alois Alzheimer}, es va topar amb uns estranys símptomes en una pacient anomenada \textit{Auguste Deter} de cinquanta-un anys. Aquesta dona patia pèrdua de memòria a curt termini i al·lucinacions auditives, cosa que va deixar perplex al doctor \textit{Alzheimer}. Descobrir els motius del comportament de la seva pacient, es va convertir en l'obsessió del doctor i, cinc anys després, quan la pacient \textit{Auguste Deter} va morir en un asil de la ciutat de \textit{Frankfurt}, \textit{Alzheimer} va conservar dues coses que van ser elements clau en la història d'aquesta malaltia. Els elements conservats pel doctor, varen ser l'historial clínic de la pacient i estudis del seu cervell. El Dr. \textit{Alzheimer} va portar a \textit{Munich}, concretament al laboratori d'\textit{Emil Kraepelin} un pioner en l'àrea psiquiàtrica, els seus descobriments amb la finalitat de poder començar una investigació encara més rigorosa.\\
Durant l'autòpsia del cervell de la seva pacient, el doctor \textit{Alzheimer} va descobrir que l'escorça cerebral era més estreta del normal i hi havia dos tipus d'anomalies notables: plaques d'amiloide, que són acumulacions de proteïnes entre les neurones, i cabdells d'una altra proteïna anomenada tau. Aquestes anomalies estan relacionades amb la disminució de la funció neuronal.\\
El doctor va presentar el cas de la seva pacient en una reunió de psiquiatria, però no va generar molt interès. No obstant això, l'any 1910, el Dr. \textit{Kraepelin} va començar a referir-se a aquella malaltia com la “malaltia d'Alzheimer”. El Dr. \textit{Alzheimer} no podia imaginar que aquell primer contacte amb aquella dona de cinquanta-un anys iniciaria una llarga i difícil batalla (batalla que avui dia encara continua amb investigadors a primera línia de batalla) per descobrir tots els símptomes i una cura per la causa de demència més comuna.\\
El Dr. \textit{Alois Alzheimer} va morir l'any 1915, però el seu llegat en el camp de la biomedicina encara és viu. El doctor és reconegut no solament per la seva descripció inicial d'una afecció, sinó també per ser un exemple d'investigador clínic. Va establir un estàndard per comprendre els desordres neurodegeneratius en mantenir una estreta relació amb els seus pacients i utilitzar eines científiques per explicar com els símptomes es relacionen amb els canvis físics del cervell.
\subsection*{Qué és l'Alzheimer i quins factors contribueixen a la seva aparició?}
\addcontentsline{toc}{subsection}{Qué és l'Alzheimer i quins factors contribueixen a la seva aparició?}
Atès que l'Alzheimer és la causa més comuna de demència, abans de definir res sobre la malaltia d'Alzheimer cal definir de manera clara que és la demència.\\
La demència, també coneguda com a trastorn neurocognitiu major, per definició, és un conjunt de símptomes que causen diverses infermetats. Aquests símptomes inclouen: afectacions en la memòria, afectacions en el comportament i afectacions en les habilitats socials de tal manera que dificulten les activitats quotidianes i la independència social.\\
Moltes de les infermetats que causen demència provoquen simptomatologia similar com pèrdua de la memòria i de l'orientació, comportament agressiu, problemes de parla i diverses afectacions a escala física. Cal remarcar que aquests símptomes es poden manifestar en moltes maneres, tot depenent de la persona afectada.\\
Un cop definit el terme de demència, es pot procedir a la definició d'Alzheimer.\\
L'Alzheimer és un tipus de demència que causa problemes amb la memòria, el pensament i el comportament. Els símptomes, generalment, es desenvolupen lentament i empitjoren amb el pas del temps, fins que són tan greus que interfereixen amb les tasques quotidianes. Cal remarcar que l'Alzheimer tot i ser la causa més comuna de demència no és l'única. En la següent imatge es pot observar un gràfic que de les demències més comunes:
\begin{figure}[H]
    \centering
    \includegraphics[scale = 0.5]{images/image_2023-03-08_172829369.png}
    \caption{Gràfic de les demències més comunes}
    \label{fig:demències}
\end{figure}
De tots els casos de demència que es diagnostiquen, s'atribueix un percentatge d'entre un 60\% i un 80\% a l'Alzheimer.\\
En cap cas aquesta malaltia és quelcom normal de l'envelliment, tot i que el factor de risc més important sigui el pas del temps, fet que implica fer-se gran. La majoria de les persones diagnosticades amb Alzheimer són majors de seixanta-cinc anys. Tot i que, cal assenyalar que la infermetat d'Alzheimer no afecta només a persones d'avançada edat, ja que al voltant de dues-centes mil persones, solament als Estats Units, menors de seixanta-cinc anys pateixen, de manera primerenca, aquesta malaltia.\\
L'Alzheimer és una infermetat progressiva. Això significa que empitjora amb el pas del temps. En les seves primeres etapes la pèrdua de memòria és lleu, però en l'etapa final, les persones perden la capacitat de mantenir una conversa i respondre a l'entorn.\\
Tenint en compte el factor de l'edat, existeixen una totalitat de quatre factors de risc.\\
El primer factor de risc, és l'herència. La malaltia pot ser hereditària on la meitat o més de cada generació està afectada per un gen dominant. En famílies on hi ha dues generacions consecutives amb membres afectats, els fills tenen un 50\% de probabilitats de patir Alzheimer, si viuen fins als vuitanta-cinc anys. En la forma esporàdica de l'infermetat el risc no es pot determinar, però augmenta si la persona té una constitució genètica amb el gen APO E-4. El risc solament augmenta en un 20\% sense la mutació del gen, un 47\% per aquells amb una mutació i un 91\% per aquelles persones amb dues mutacions del gen.\\
El segon factor de risc que pot contribuir a l'aparició d'Alzheimer, és un traumatisme cranial. Cal remarcar que els danys cerebrals en un pacient d'Alzheimer són majors, però el fet de patir un traumatisme cranial pot ser un factor desencadenant de la malaltia, tot i que la majoria de traumatismes cranials no són Alzheimer ni desencadenen la infermetat.
Un altre factor molt important a tenir en compte és el sexe. Com es veurà en la posteriorment a la secció d'estadístiques d'afectats per la infermetat, les dones són molt més propenses a tenir Alzheimer que els homes. De fet, s'estima només a Espanya, que en el 69\% dels casos d'Alzheimer el sexe afectat, és el sexe femení.\\
Com s'ha comentat anteriorment, els anys de vida és el factor més influent, però aquest deixa de ser un element dominant a partir dels noranta anys. A partir d'aquesta edat, el risc de patir Alzheimer disminueix, normalment. És capital posar en rellevància que, en cap cas, el nivell educatiu o intel·lectual, l'exposició continuada a l'alumini, els virus lents, les infeccions o viure en una zona rural o urbana són considerats factors de risc.
\subsection*{Quina simptomatologia ens ha de posar en alerta?}
\addcontentsline{toc}{subsection}{Quina simptomatologia ens ha de posar en alerta?}
Cent vint-i-dos anys després del descobriment de l'infermetat d'Alzheimer, per part del Dr. \textit{Alois Alzheimer}, la batalla per descobrir tots els símptomes de la malaltia continua. Avui en dia, encara no es coneix tota la simptomatologia que presenta la malaltia d'Alzheimer, però sí que hi ha determinats signes que han de posar en alerta a una persona i als seus familiars i amics.\\
Ara per ara, estan establerts deu símptomes que són senyals d'alarma per una possible demència per Alzheimer.\\
\textbf{La primera alarma és la pèrdua de memòria}. Especialment en els primers estadis de la infermetat, un dels principals senyals és oblidar informació recent, oblidar dates o esdeveniments importants, demanar la mateixa informació de manera reiterada, etc.\\
\textbf{El segon senyal d'alarma és la dificultat per planificar o resoldre problemes}. Algunes persones poden experimentar problemes per desenvolupar i seguir una rutina, treballar amb nombres seguir una recepta de cuina, concentrar-se en una tasca, etc.\\
\textbf{Com a tercera alarma, hi ha la dificultat per desenvolupar tasques habituals ja sigui a casa, a la feina o al temps d'oci}. Per exemple, es pot tenir dificultat per tasques tan usuals com rentar-se la cara, arribar d'un punt “A” a un punt “B”, sent el punt “B” una localització coneguda per la persona, administrar un pressupost o recordar les normes d'un joc.\\
\textbf{La quarta alerta és la desorientació en l'espaitemps}. Les persones amb Alzheimer obliden dates importants, les estacions de l'any i el pas del temps en general. També és molt probable que s'oblidin del lloc on es troben en aquell moment i de com han arribat allí.\\
\textbf{El cinquè senyal d'alarma és la dificultat de comprensió visual}. La comprensió visual es refereix a entendre imatges visuals i com els objectes es relacionen entre ells a l'entorn. Manca de comprensió visual també inclou la dificultat per llegir, jutjar distàncies, determinar el contrast de colors, etc.\\
\textbf{La sisena alarma és tenir impediments per utilitzar el llenguatge}. Tant en el llenguatge escrit com en la parla, les persones amb Alzheimer poden tenir dificultats per a seguir una conversa o que, per exemple, a mitja conversa parin sense tenir ni la més mínima idea del que estaven dient o que repeteixin reiteradament quelcom que acaben de dir fa escassos instants. També pot ser que no puguin trobar les paraules adients o que anomenin les coses per un nom incorrecte.\\
\textbf{L'alerta número set és col·locar objectes en llocs diferents de l'habitual i tenir dificultats per trobar-los}. Una persona afectada per la malaltia sol posar coses fora del seu lloc essent després completament incapaç de trobar-les. De vegades, és possible, que acusin altres persones de robar-los.\\
\textbf{L'antepenúltim senyal d'alarma és la disminució o la falta de judici}. La disminució de l'activitat cerebral pot provocar alteracions en la capacitat de jutjar el perill que comporta una acció i també pot provocar una alteració en la capacitat de prendre decisions.\\
\textbf{Com a penúltima alerta es troba la pèrdua d'iniciativa}. Cal subratllar, que aquest senyal és un “dels menys importants”, ja que la pèrdua d'iniciativa per participar en passatemps, activitats socials, projectes de treball o esports també es pot donar en altres malalties completament diferents de l'Alzheimer, com per exemple, la depressió.\\
\textbf{L'última alarma són els canvis d'humor o de personalitat}. Les persones amb la malaltia d'Alzheimer poden arribar a sospitar de tothom, ser persones agressives, temoroses o ansioses.
\subsection*{Quines fases té la malaltia?}
\addcontentsline{toc}{subsection}{Quines fases te la malaltia?}
En tot el món, professionals i cuidadors utilitzen l'escala de deteriorament global, desenvolupada pel Dr. Barry Reisberg, director del programa d'educació i investigació de la infermetat d'Alzheimer. Aquesta escala serveix per determinar en quin nivell de demència en què es troba la persona afectada d'Alzheimer.\\
Dividida en dos grans blocs; predemència (estadis 1 a 3) i demència (estadis 4 a 7), l'escala consta de set estadis clínics diferents.\\
El punt d'inflexió on la persona ja no pot viure sense assistència continuada, és l'estadi 5.
\subsubsection*{Llistat d'estadis}
\addcontentsline{toc}{subsubsection}{Llistat d'estadis}
\begin{itemize}
    \item Bloc predemència
    \begin{itemize}
        \item Estadi 1: No demència observable.
        \item Estadi 2: Pèrdua de memòria relacionada amb l'edat.
        \item Estadi 3: Deteriorament cognitiu lleu.
    \end{itemize}
    \item Bloc de demència
    \begin{itemize}
        \item Estadi 4: Declivi cognitiu moderat. Demència lleu.
        \item Estadi 5: Declivi cognitiu moderadament sever. Demència moderada.
        \item Estadi 6: Declivi cognitiu sever. Demència moderadament severa
        \begin{itemize}
            \item Estadi 6A.
            \item Estadi 6B.
            \item Estadi 6C.
            \item Estadi 6D.
            \item Estadi 6E.
        \end{itemize}
        \item Estadi 7: Declivi cognitiu molt sever. Demència severa.
        \begin{itemize}
            \item Estadi 7A.
            \item Estadi 7B.
            \item Estadi 7C.
            \item Estadi 7D.
            \item Estadi 7E.
            \item Estadi 7F.
        \end{itemize}
    \end{itemize}
\end{itemize}
\subsubsection*{Característiques de cada estadi}
\addcontentsline{toc}{subsubsection}{Característiques de cada estadi}
\textbf{\underline{Estadi 1}:}\\
Aquest primer estadi on no es pot observar cap mena de demència és l'estadi que es dona a qualsevol edat i és l'estadi en el qual no es presenta cap símptoma de demència. Aquest és l'anomenat estadi de normalitat.\\
\textbf{\underline{Estadi 2}:}\\
El segon estadi, és l'estadi de les pèrdues de memòria relacionades amb l'edat. Les persones de seixanta-cinc anys o més asseguren tenir dificultats cognitives i/o funcionals. Aquestes dificultats inclouen recordar, com ho feien abans, els noms, dates, on han posat un determinat objecte, etc. En el món de la investigació clínica, s'han proposat molts noms per denominar aquesta fase, però el més acceptat és el deteriorament cognitiu subjectiu. Com s'ha demostrat, científicament, que els símptomes d'aquest estadi no són notables per observadors externs, per tant, aquest segon estadi són el que s'anomena “coses de l'edat”. Si la demència ha de continuar avançant, ho farà en un període d'aproximadament quinze anys.\\
\textbf{\underline{Estadi 3}:}\\
Aquesta etapa de la malaltia, anomenada deteriorament cognitiu lleu (MCI), es caracteritza per tenir dificultats subtils en la memòria i altres habilitats mentals, que són detectades per persones properes a la persona afectada. Aquests símptomes poden incloure dificultat per planificar esdeveniments socials complexos, disminució del rendiment laboral o problemes per aprendre habilitats noves. És important buscar ajuda mèdica com abans millor per determinar si aquests símptomes són causats per l'Alzheimer o altres condicions. El pronòstic d'aquest estadi és variable, però la duració mitjana és d'uns set anys.\\
La gestió de les persones que es troben en aquest estadi inclou l'assessorament sobre la convivència de continuar en un treball exigent i, de vegades, pot incloure una “retirada estratègica” en forma de jubilació per reduir l'estrès i l'ansietat.\\
\textbf{\underline{Estadi 4}:}\\
En aquest quart estadi de la malaltia d'Alzheimer, anomenat declivi cognitiu moderat o demència lleu, el dèficit més comú, és la disminució de la capacitat per dur a terme activitats quotidianes, cosa que pot dificultar la independència. Els símptomes de pèrdua de memòria també es fan evidents per als observadors externs. La incapacitat de recordar esdeveniments recents importants i errors en recordar el dia de la setmana o l'estació de l'any són les primeres pèrdues.\\
Tot i aquests dèficits, les persones en aquest estadi encara poden viure de manera independent en entorns comunitaris. L'estat d'ànim predominant en aquest estil és l'aplanament de l'afecte, el retraïment i la negació del dèficit de memòria també és un símptoma molt comú. El diagnòstic d'Alzheimer es pot fer amb certesa des d'aquest estadi, que dura aproximadament dos anys.\\
\textbf{\underline{Estadi 5}:}\\
En aquest cinquè estadi de declivi cognitiu moderadament sever o de demència moderada, els dèficits són prou significatius per a impedir la vida independent de la persona malalta i, evitar així, catàstrofes. Això es manifesta en una disminució de la capacitat per, per exemple, escollir la roba adient a les condicions meteorològiques. També es disminueix la capacitat d'afrontar les circumstàncies de la vida diària. La persona malalta d'Alzheimer ja no pot cuidar de si mateixa i, per tant, requereix ajuda per coses tan bàsiques com menjar o cuidar les finances. Un altre parell d'aspectes que, normalment, es veuen compromesos són la seva orientació i la memòria.\\
\textbf{\underline{Estadi 6}:}\\
El sisè estadi anomenat declivi cognitiu sever o demència moderadament severa, les habilitats per dur a terme activitats quotidianes són limitades. Es poden identificar, com s'ha vist en el llistat d'estadis, cinc subestadis successius pel que fa a la funcionalitat.\\
Els pacients en l'estadi 6A, a més de no poder escollir la seva roba sense ajuda, comencen a requerir assistència per vestir-se adequadament. Si no estan supervisats, poden posar-se la roba del revés o poden tenir problemes per posar els braços a les mànigues correctes. Aquest sisè estadi de demència moderadament severa (6A a 6E) durà aproximadament dos anys i mig.\\
En un punt similar en l'avanç de l'Alzheimer, estan els pacients que es troben en l'estadi 6B. Aquestes persones perden la capacitat de banyar-se sense assistència. Una de les dificultats més comunes en aquest estadi, és regular la temperatura de l'aigua. Encara que el cuidador pot ajustar la temperatura, el malalt d'Alzheimer encara pot banyar-se per si sol. Tot i que, a mesura que avança la malaltia, es presenten més dificultats per banyar-se i vestir-se de forma independent, així com problemes addicionals en altres àrees de la higiene personal, com raspallar-se les dents.\\
En els tres últims subestadis, a més de tenir dificultats amb el bany, la persona afectada d'Alzheimer, pot oblidar treure's la roba per banyar-se o estirar la cadena després d'anar al lavabo. A més, en aquest estadi, la persona tendeix a patir incontinència urinària i fecal. Malgrat això, la incontinència pot tractar-se o fins i tot prevenir-se en alguns casos mitjançant estratègies addicionals per manejar la incontinència, com la roba de llit i la roba interior absorbent.\\
Els dèficits cognitius en aquesta etapa també són molt greus, el que fa que les persones amb Alzheimer tinguin moltes dificultats per recordar dates importants de les seves vides, com l'adreça de casa o les condicions climàtiques del dia. Sovint, confonen els seus éssers estimats amb altres persones o tenen dificultats per identificar correctament els membres de la seva família. A la fi d'aquesta etapa, la capacitat de parlar també es veu afectada. A més, els records d'esdeveniments actuals són, generalment, defectuosos i les persones amb Alzheimer sovint no poden anomenar líders polítics rellevants o recordar esdeveniments significatius de la seva vida. En aquest estadi, també poden tenir dificultats per executar tasques matemàtiques bàsiques.\\
Els canvis emocionals també són comuns en aquest estadi, o  poden incloure inquietud, comportament sense propòsit o inapropiat i  arravataments verbals. A més, a causa de la seva incapacitat per sobreviure de forma independent, les persones amb Alzheimer desenvolupen la por a quedar-se soles. El tractament d'aquests símptomes conductuals i psicològics pot incloure assessorament i intervencions farmacològiques.\\
\textbf{\underline{Estadi 7}:}\\
El setè i últim estadi, és l'estadi del declivi cognitiu molt sever o l'estadi de demència severa. En aquest estadi, les persones necessiten assistència contínua amb les activitats bàsiques de la vida diària. Es poden identificar sis subetapes consecutives al llarg d'aquesta etapa final. Al principi d'aquesta etapa, el discurs se circumscriu de tal manera que es limita a aproximadament mitja dotzena de paraules intel·ligibles o menys. A mesura que avança aquesta etapa, la parla es limita com a màxim a una sola paraula intel·ligible. A l'estadi 7C, l'individu perd la capacitat de deambular i asseure's de manera independent i, a l'etapa 7E perd la capacitat de somriure. A l'estadi final 7F, l'individu perd la capacitat d'aixecar el cap de manera independent.\\
Moltes persones amb Alzheimer sucumbeixen en diversos moments de l'estadi 7 a pneumònia, ulceracions infectades o altres afeccions. En aquest estadi també es fan evidents canvis físics i neurològics, com la rigidesa física i les contractures, que impedeixen el moviment de les articulacions. També apareixen reflexos neurològics infantils o primitius que estan presents en un nadó, però que desapareixen en el desenvolupament normal.
\subsection*{De quines maneres es pot prevenir l'Alzheimer?}
\addcontentsline{toc}{subsection}{De quines maneres es pot prevenir l'Alzheimer?}
Segons CEAFA, la confederació espanyola d'Alzheimer, hi ha nou formes de reduir el risc de patir Alzheimer.\\
El primer consell és suar, atès que l'exercici cardiovascular que augmenta el ritme cardíac i el flux de sang al cervell i al cos està associat amb una disminució del risc de deteriorament cognitiu.\\
Com a segon mètode preventiu es troben els desafiaments mentals. La formació en qualsevol etapa de la vida és beneficiosa per la salut mental, sigui en línia, sigui en una institució educativa. Fins i tot les activitats mentals com resoldre trencaclosques, jugar a cartes o prendre classes d'art tenen un impacte positiu.\\
El tercer consell, encara que és beneficiós per la salut en general, també és un bon preventiu contra l'Alzheimer. Aquest consell és deixar de fumar. Si una persona deixa de fumar, completament, pot arribar a reduir el risc de patir l'infermetat als mateixos nivells que una persona que no ha fumat mai.\\
Parar compte amb la salut general, tot i que no pugui ser a primera vista, factors com l'obesitat, el colesterol i la hipertensió a part d'augmentar el risc d'infermetats cardíaques també augmenta el risc de demència. Per tant, el quart consell de CEAFA, seria portar un control adequat de la salut.\\
Utilitzar un casc per dur a terme segons quins esports, posar-se el cinturó al cotxe i evitar les caigudes són mesures de seguretat per evitar lesions cerebrals que poden augmentar el risc de deteriorament cognitiu. Per tant, el consell número cinc és protegir de manera adequada el cap.\\
El sisè consell que CEAFA posa a disposició de les persones és portar una dieta sana i equilibrada. S'ha demostrat que la dieta mediterrània i la dieta mediterrània DASH (mètodes dietètics per detenir la hipertensió) poden ajudar a reduir el risc d'Alzheimer. Els beneficis d'aquestes dietes es deuen al fet que són dietes que contenen una gran quantitat de verdures crucíferes i bulbs, verdures de fula verda, oli d'oliva, fruites, nous, cacau, cafè, peixos rics en omega-3 i omega-6 làctics desnatats, menys sal i menys alcohol.\\
Infermetats com l'apnea i l'insomni poden causar problemes de memòria i pensament. Per tant dormir suficient és un bon consell de prevenció.\\
També és un factor reductor del risc el fet de portar una vida social activa. Prestar ajuda a qui ho necessita, fer exercici amb companyia d'amistat, dur a terme activitats d'oci amb familiars i amics, etc. són alguns exemples de vida social activa.\\
El novè, i últim consell, és la reducció de l'estrès. Alguns estudis associen una història de depressió amb una major probabilitat de disminució cognitiva, per la qual cosa és important buscar ajuda de professionals per a tractar la depressió, l'ansietat, l'estrès i altres problemes de salut mental.
\subsection*{Quin és el mètode diagnòstic actual?}
\addcontentsline{toc}{subsection}{Quin és el mètode diagnòstic actual?}
Hi ha persones que no reconeixen que tenen un problema quan es manifesten pèrdues de memòria o altres senyals d'alerta d'Alzheimer. Molts cops aquests símptomes es fan més evidents per la gent de l'entorn com poden ser amics o familiars.\\
El primer que cal fer en el seguiment de la simptomatologia és, cercar un metge amb el qual la persona se senti còmoda. De fet, no hi ha un sol tipus de metge que la seva especialitat sigui la diagnosi i el tractament de l'Alzheimer. La gran majoria de persones es posen en contacte, en primera instància, amb el metge de capçalera per poder posar de manifest les seves preocupacions. Tot i que, sovint, els metges de capçalera controlen el diagnòstic diferencial per si mateixos, també poden requerir l'ajuda d'especialistes, tals com: neuròlegs, psiquiatres i psicòlegs.\\
La diagnosi de l'Alzheimer, no es basa en els resultats d'una sola prova, sinó que es realitza un reconeixement complet del pacient per avaluar la salut general i identificar quines poden ser les causes de l'alteració del funcionament normal de la ment. Un cop descartades altres malalties, la persona encarregada de donar el diagnòstic diferencial final pot determinar si realment es tracta d'Alzheimer o si es tracta d'un altre tipus de demència. Per tant, abans de donar com a segur que la persona pateix la malaltia d'Alzheimer, s'han de seguir una sèrie de passos i un cop fetes les proves pertinents caldrà analitzar amb molt detall els seus resultats.\\
El primer dels passos és entendre el problema, és a dir, s'ha de donar resposta a una sèrie de qüestions com, quins símptomes s'han patit? Quan van començar a manifestar-se? Amb quina freqüència es manifesten? Han anat a pitjor en cada manifestació?\\
El segon pas és dur a terme una revisió completa de l'historial clínic. En aquest pas, el metge citarà tant a la persona que se sotmet a les proves com a altres persones properes al pacient per tal de poder reunir tota la informació possible sobre infermetats mentals i físiques tant actuals com passades. El metge també sol·licitarà els antecedents de les malalties de la família, especialment si algun dels membres ha patit o pateix Alzheimer o alguna classe de demència.\\
Una avaluació de l'estat d'ànim i de l'estat mental serà el proper pas abans de donar l'Alzheimer com a diagnòstic. En aquesta avaluació, es realitzen una sèrie de proves que estan orientades a avaluar la memòria, la capacitat de resoldre problemes senzills i altres tipus d'habilitats. Aquest test dona al metge una idea general de l'estat de consciència del pacient. És a dir, si és conscient de la simptomatologia que està patint, si coneix la data, l'hora i on està, si pot recordar una petita llista de paraules, seguir ordres i fer càlculs matemàtics senzills. Una cosa molt comuna és, preguntar al pacient, coses com la seva direcció, l'any, qui és el president del govern, que va sopar el dia anterior, etc. Després de fer aquestes proves, el metge avaluarà si el pacient pateix depressió o alguna altra afecció que pot causar pèrdues de memòria. Un cop s'hagi fet l'avaluació de l'estat d'ànim i de l'estat mental, serà moment de dur a terme un examen físic i una sèrie de proves de diagnòstic. En aquest punt, el metge avaluarà la dieta i la nutrició, efectuarà una revisió de la pressió arterial, la temperatura i el pols, auscultarà el cor i els pulmons i executarà altres procediments per avaluar la salut general del pacient. També es recol·lectaran mostres de sang i d'orina per identificar malalties com anèmia, infeccions, diabetis, insuficiència renal, infermetats hepàtiques, deficiències vitamíniques, anomalies a la tiroide, problemes cardíacs, problemes dels vasos sanguinis o problemes pulmonars. Els diferents quadres clínics que tenen els símptomes anteriorment descrits poden ser confosos amb els símptomes de la demència.\\
L'últim pas abans de donar l'Alzheimer com a diagnòstic final, és portar a cap un examen neurològic. En aquesta prova es faran proves per verificar si la persona pateix trastorns cerebrals diferents de la malaltia d'Alzheimer. El metge també avaluarà l'estat dels reflexos, la coordinació, el moviment dels ulls, la parla i la sensació. Abans de concretar un diagnòstic, però, també se cercaran evidències d'accidents vasculars cerebrals, Parkinson, tumors cerebrals, acumulació de líquids al cervell i altres infermetats que poden afectar a la memòria. Normalment, aquests estudis inclouen imatges de \textit{MRI} o \textit{CT scan}. L'estudi de les imatges resultants d'aquestes proves poden revelar la presència de tumors, d'accidents vasculars cerebrals, danys causats per un trauma al cap o acumulació de líquid.

\subsection*{Quin és el tractament actual?}
\addcontentsline{toc}{subsection}{Quin és el tractament actual?}
Actualment, hi ha molt pocs medicaments aprovats per tractar específicament la infermetat d'Alzheimer. A causa del fet que encara no es coneixen, completament, les causes d'aquesta malaltia, no hi ha medicines que puguin curar o prevenir aquesta forma de demència. Les primeres medicacions que es van desenvolupar per tractar aquesta malaltia van ser els inhibidors de la colinesterasa i la memantina. Aquests medicaments poden endarrerir l'avanç de la simptomatologia, però no poden tractar les lesions cerebrals subjacents ni prolongar la vida dels pacients.\\
Hi ha dos tipus de medicaments: els que alleugen temporalment els símptomes i els que retarden la malaltia. No obstant això, no són efectius en tots els casos i poden perdre la seva eficàcia amb el temps.\\
Els medicaments per la malaltia d'Alzheimer estan aprovats per etapes determinades d'aquesta. Aquestes etapes es basen en els resultats de les proves que avaluen la memòria, la consciència de l'espai-temps, el pensament i el raonament.\\
No obstant això, els metges poden receptar medicines per a tractar la infermetat d'Alzheimer per etapes diferents de les aprovades per les institucions públiques (A Espanya l'AEMPS). Les etapes de la malaltia no són precises, les respostes individuals als medicaments, variables i les opcions de tractament, limitades.\\
Les medicines aprovades, avui dia, per l'Alzheimer estan pensades per a l'estadi de deteriorament cognitiu lleu. És a dir, són medicines pensades per tractar els signes de la demència en general, no per l'Alzheimer, perquè cal remarcar que en la fase de deteriorament cognitiu lleu, encara no es pot assegurar que la malaltia sigui l'Alzheimer.\\
\subsubsection*{Inhibidors de la colinesterasa}
\addcontentsline{toc}{subsubsection}{Inhibidors de la colinesterasa}
La malaltia d'Alzheimer afecta al cervell reduint els nivells d'acetilcolina, neurotransmissor important per la memòria, la consciència i el pensament. Els inhibidors de colinesterasa funcionen en augmentar la quantitat d'acetilcolina disponible en les cèl·lules nervioses en prevenir la descompensació. No obstant això, els medicaments no poden curar la infermetat ni detenir la destrucció de les cèl·lules nervioses i amb el temps perden efectivitat. Alguns efectes secundaris comuns inclouen nàusees, vòmits i diarrea, però poden ser reduïts amb una dosi baixa a l'inici del tractament i prenent els medicaments amb els aliments.\\
És important tenir en compte que els inhibidors de la colinesterasa no es recomanen per a persones amb arrítmies cardíaques.
\subsubsection*{Memantina per a estadis més avançats}
\addcontentsline{toc}{subsubsection}{Memantina per a estadis més avançats}
La memantina és l'única medicina aprovada per a etapes avançades de la malaltia d'Alzheimer. Aquest medicament controla l'activitat del glutamat, un neurotransmissor clau en l'aprenentatge i la memòria. Els efectes adversos més comuns són: mareigs, mals de cap, confusió i exaltació.
\subsubsection*{Aducanumab}
\addcontentsline{toc}{subsubsection}{Aducanumab}
L'aducanumab és una teràpia intravenosa recentment aprovada per la FDA (l'agència de medicaments dels EUA) per a pacients amb deteriorament cognitiu lleu i demència lleu per l'infermetat d'Alzheimer. S'ha aprovat mitjançant una disposició d'aprovació accelerada a causa de la seva capacitat de reduir la proteïna beta-amiloide, considerada un factor important en el transcurs de la malaltia. No obstant això, el seu efecte en el rendiment diari, la memòria o el raonament no està clar.\\
Alguns efectes secundaris poden incloure anomalies en les imatges cerebrals relacionades amb l'amiloide. Anomalies com un edema cerebral, dipòsits  d'hemosiderina o microhemorràgies. Aquests canvis s'han de monitorar amb imatges repetides de ressonància magnètica.
\subsubsection*{Lecanemab}
\addcontentsline{toc}{subsubsection}{Lecanemab}
El medicament Lecanemab, ha mostrat resultats prometedors en pacients amb una forma lleu de la malaltia d'Alzheimer i deteriorament cognitiu per culpa d'aquesta. la FDA estima que aquest 2023 estigui disponible per al públic.\\
Un assaig clínic de fase 3 va demostrar que el Lecanemab va reduïr en un 27\% el deteriorament cognitiu en pacients amb Alzheimer primerenc. Actua evitant la formació de plaques amiloides en el cervell. Aquest estudi és el més gran dut a terme fins ara per avaluar si l'eliminació de les plaques amiloides pot endarrerir l'avenç de l'Alzheimer.\\
La FDA està avaluant el Lecanemab. A més, s'està duent a terme un altre estudi per a determinar la seva eficàcia en persones de risc de desenvolupar Alzheimer, incloent aquelles amb familiars propers afectats per la malaltia.
\subsection*{Estadístiques d'afectats en l'àmbit nacional i autonòmic}
\addcontentsline{toc}{subsection}{Estadístiques d'afectats a en l'àmbit nacional i autonòmic}
A continuació, es presenten les dades enregistrades per CEAFA en l'estudi que va realitzar l'any 2019 dels casos diagnosticats d'Alzheimer a Espanya.
\begin{table}[h]
    \centering
    \begin{tabular}{ |c | c | c | c | c| } 
        \hline
        \hline Comunitat autònoma & Nº casos & Població & \% sobre la població & \% sobre casos\\
        \hline
         Andalusia & 109.311 & 8.247.404 & 1,3\% & 23,8\% \\
         \hline
         Aragó & 17.568 & 1.320.586 & 1,3\% & 3,8\% \\
         \hline
         Astúries & 9.481 & 1.022.205 & 0,9\% & 2,1\% \\
         \hline
         Illes Balears & 10.917 & 1.188.220 & 0,9\% & 2,4\% \\
          \hline
         Illes Canàries & 19.309 & 2.206.901 & 0,9\% & 4,2\% \\
          \hline
         Cantabria & 3.864 & 581.641 & 0,7\% & 0,8\% \\
          \hline
         Castella i Lleó & 19.738 & 2.407.733 & 0,8\% & 4,3\% \\
          \hline
         Castella la Manxa & 12.600 & 2.034.877 & 0,6\% & 2,7\% \\
          \hline
         Catalunya & 46.949 & 7.566.430 & 0,6\% & 10,2\% \\
          \hline
         Comunitat Valenciana & 91.673 & 4.974.969 & 1,8\% & 20\% \\ 
          \hline
         Extremadura & 3.776 & 1.065.424 & 0,4\% & 0,8\% \\
          \hline
         Galícia & 25.572 & 2.700.441 & 0,9\% & 5,6\% \\ 
          \hline
         Comunitat de Madrid & 47.276 & 6.641.648 & 0.7\% & 10,3\% \\ 
         \hline
         Regió de Múrica & 12.520 & 1.487.663 & 0,8\% & 2,7\% \\
          \hline
         Comunitat foral de Navarra & 5.951 & 649.946 & 0,9\% & 1,3\% \\ 
          \hline
         Euskal Herria & 20.031 & 2.177.880 & 0,9\% & 4,4\% \\ 
          \hline
         La Rioja & 2.331 & 313.571 & 0,7\% & 0,5\& \\ 
          \hline
         Ceuta & 0 & 84.829 & 0\%  & 0\% \\ 
          \hline
         Melilla & 0 & 84.689 & 0\%  & 0\% \\
          \hline
        Total & 458.869 & 49.937.060 & 1\% & 100\% \\
          \hline
    \end{tabular}
    \caption{Distribució territorial dels casos enregistrats de persones amb demència l'any 2019}
    \label{tab:taula1}
\end{table}

Seguidament i per tenir una visió una mica més gràfica, es posaran sobre el mapa les dades de la taula anterior. Com es pot observar hi ha dues representacions diferents de les dades sobre el mapa d'Espanya. La primera representació és, la coloració del mapa segons el tant per cent de casos respecte a la població de cada comunitat autònoma. La segona representació és, la coloració del mapa d'Espanya respecte al nombre de casos d'Alzheimer per comunitat autònoma.
\begin{figure}[H]
    \centering
    \includegraphics[scale = 0.6]{images/distribucio territorial en base a percentatge.jpg}
    \caption{Coloració geogràfica dels casos enregistrats de persones amb Alzheimer l'any 2019 segons el tant per cent respecte a la població de cada comunitat autònoma.}
    \label{fig:colorpercent}
\end{figure}
\begin{figure}[H]
    \centering
    \includegraphics[scale = 0.6]{images/distribucio territorial en base a casos.jpg}
    \caption{Coloració geogràfica dels casos enregistrats de persones amb Alzheimer l'any 2019 segons el nombre de casos.}
    \label{fig:colorcasos}
\end{figure}
Si es fa una comparació entre els dos mapes, s'observa que la coloració de moltes de les comunitats autònomes canvia, perquè no totes les comunitats autònomes tenen el mateix nombre d'habitants.\\
Abans de finalitzar aquesta secció, és necessari donar les dades de la taula \ref{tab:taula1} segregades per sexe i comunitat autònoma per tenir una idea encara més clara de com es distribueix la malaltia en el regne d'Espanya.
\begin{table}[h]
    \centering
    \begin{tabular}{ |c | c | c | c | c | c | } 
        \hline
        \hline Comunitat autònoma & \multicolumn{2}{|c|}{Home} & \multicolumn{2}{|c|}{Dona} & Total\\
        \hline
        Andalusia & 32.817 & 30\% & 76.494 & 70\% & 109.311\\
        \hline
        Aragó & 5.830 & 33,2\% & 11.739 & 66,8\% & 17.568\\
        \hline
        Astúries & 2.556 & 27\% & 6.924 & 73\% & 9.481\\
        \hline
        Illes Balears & 3.318 & 30,4\% & 7.958 & 69,6\% & 10.917\\
        \hline
        Illes Canàries & 6.169 & 31,9\% & 13.139 & 68\% & 19.309\\
        \hline
        Cantàbria & 1.024 & 26,5\% & 2.840 & 73,5\% & 3.864\\
         \hline
        Castella i Lleó & 5.611 & 28,4\% & 14.128 & 71,6\% & 19.738\\
         \hline
        Castella la Manxa & 4.091 & 32,5\% & 8.509 & 67,5\% & 12.600\\
         \hline
        Catalunya & 13.952 & 29,7\% & 32.998 & 70,3\% & 46.949\\
         \hline
        Comunitat Valenciana & 29.395 & 32,1\% & 62.278 & 67,9\% & 91.673\\
         \hline
        Extremadura & 1.231 & 32,6\% & 2.545 & 67,4\% & 3.776\\
         \hline
        Galícia & 7.161 & 28\% & 18.411 & 72\% & 25.572\\
         \hline
        Comunitat de Madrid & 15.586 & 30,9\% & 32.689 & 69,1\% & 47.276\\
         \hline
        Regió de Múrcia & 4.265 & 34,1\% & 8.256 & 65,9\% & 12.520\\
         \hline
        Comunitat foral de Navarra & 1.755 & 29,5\% & 4.196 & 70,5\% & 5.951\\
         \hline
        Euskal Herria & 5.719 & 28,6\% & 14.312 & 71,4\% & 20.031\\
         \hline
        La Rioja & 767 & 32,9\% & 1.564 & 67,1\% & 2.331\\
         \hline
        Ceuta & 0 & 0\% & 0 & 0\% & 0\\
        \hline
        Melilla & 0 & 0\% & 0 & 0\% & 0\\
        \hline
        Total & 140.248 & 30,6\% & 318.621 & 69,4\% & 458.869\\
        \hline
    \end{tabular}
    \caption{Distribució territorial i per sexe dels casos enregistrats de persones amb demència l'any 2019}
    \label{tab:taula2}
\end{table}
\newpage
\subsection*{Principals associacions de lluita contra l'Alzheimer}
\addcontentsline{toc}{subsection}{Principals associacions de lluita contra l'Alzheimer}
A escala de tota Espanya hi ha moltes associacions sense ànim de lucre que es dediquen a lluitar per trobar un remei a aquesta malaltia que cada dia afecta a més gent. Com és molt difícil poder anomenar totes les associacions del país, només es llistaran les més importants.
\begin{itemize}
    \item \href{https://fpmaragall.org/ca/}{\underline{Fundació Pasqual Maragall}}
    \item \href{https://www.ceafa.es/es}{\underline{Confederació espanyola de familiars malalts d'Alzheimer i altres demències}}
    \item \href{http://www.alzfae.org/}{\underline{Fundació Alzheimer Espanya}}
    \item \href{https://www.afav.org/}{\underline{Associació familiars Alzheimer València}}
    \item \href{https://www.fevafa.org/quienes-somos/}{\underline{Federació valenciana d'associacions de familiars i amics de persones amb Alzheimer}}
    \item \href{https://www.fafac.cat/}{\underline{Federació catalana Alzheimer}}
    \item \href{http://www.lleidaparticipa.cat/index_web.php?idwc=czoxNToiYWx6aGVpbWVybGxlaWRhIjs=}{\underline{Associació de familiars de malalts d'Alzheimer i altres demències de Lleida}}
    \item \href{https://afamur.es/que-es-afamur/}{\underline{\textit{Asociación de familiares enfermos de Alzheimer de la Región de Múrcia}}}
    \item \href{https://fafal.org/}{\underline{\textit{Federación de asociaciones de familiares enfermos de Alzheimer de la CM}}}
    \item \href{https://www.afacayle.es/}{\underline{\textit{Federación regional de asociaciones de familiares de enfermos de Alzheimer de CyL}}}
    \item \href{https://www.alzheimersevilla.com/}{\underline{\textit{Asociación de familiares de enfermos de Alzherimer Santa Elena}}}
    \item \href{https://afaga.com/es/asociacion/historia/}{\underline{\textit{Asociación de familiares de enfermos de Alzherimer y otras demencias de Galicia}}}
    \item \href{https://www.asociacionalzheimer.com/}{\underline{\textit{Asociación Alzheimer Asturias}}}
\end{itemize}
\subsection*{Experiència personal amb la malaltia}
\addcontentsline{toc}{subsection}{Experiència personal amb la malaltia}
La meva àvia paterna va morir l'any 2008, des d'aleshores, el meu avi no va tornar a ser el mateix... Va començar a oblidar algunes coses que ell feia de manera diària, on posava segons quines coses, alguna data important, etc. Mentre això no anava a més, a la família no ens preocupava el més mínim aquesta simptomatologia atès que la mort de la meva àvia, la seva dona, va ser el desencadenant perquè el meu avi fos diagnosticat amb depressió. Malgrat això, quan els símptomes es varen anar agreujant, la família ens vam adonar que alguna cosa estranya passava i després d'una sèrie de proves clíniques, el meu avi va ser diagnosticat amb Alzheimer.\\
Va ser un moment molt dur per la família. Veure que una algú que, durant tota la vida, havia estat molt activa, de fer bromes, de fer plans familiars, cuinar, etc. ara ja només podia anar en caiguda lliure cap a la desconnexió mental total. A mesura que passava el temps, l'actitud del meu avi també va canviar. Al principi, no se'n recordava de dutxar-se, després va començar a oblidar on posava les coses (fet que el va portar a acusar a la família de robar-lo) i va desenvolupar una obsessió molt forta pels diners. Quan la situació es va fer insostenible, el meu pare i les seves dues germanes, amb el diagnòstic d'Alzheimer a la mà varen anar al jutjat a demanar la incapacitació total del meu padrí. Després de revisar l'informe mèdic i fer-li algunes preguntes, la jutgessa va decretar que des d'aquell precís moment, la signatura del meu avi deixava de tenir qualsevol validesa legal.\\
Mentre la família esperàvem a la concessió d'una plaça pública en una residència, el meu pare i les seves germanes decidiren internar al meu avi en una residència privada. En aquesta residència, feia tota mena d'activitats per potenciar la memòria, però maluradament cap va tenir el més mínim efecte. Al cap de poc temps d'estar en la residència, les visites cada cop es feien més pesades, tristes i difícils, ja que el meu avi va començar a confondre a la família amb altres persones. Per exemple, jo, en comptes del seu nét, era el seu germà. El meu pare, en comptes del seu fill, era el seu pare i les meves tietes les seves cosines. Amb altres membres de la família, com poden ser la meva mare i els meus cosins, la situació era idèntica.\\
Passats uns anys, la plaça pública va arribar. El meu avi va ser traslladat a una residència d'Alcarràs. La família teníem una petita esperança que el canvi de residència li dones, al meu avi, una mica d'aire i tornés per uns dies al seu ser. Lluny d'aquest desig, el meu avi continuava en caiguda lliure cap a la mort en vida. Poc abans de morir, no era capaç ni d'aixecar-se de la cadira i fer un sol pas sense anar agafat del braç. Els dies previs a la seva mort, ni tan sols es llevava del llit. Els metges decidiren administrar-li morfina i a les sis de la tarda del 17 de novembre de l'any 2015, l'àngel de la mort aparegué a l'habitació i s'endugué la seva ànima.
\section*{Marc teòric computacional}
\addcontentsline{toc}{section}{Marc teòric computacional}
\subsection*{Qué entenem per intel·ligència artificial?}
\addcontentsline{toc}{subsection}{Que entenem per intel·ligència artificial?}
La història de la intel·ligència artificial neix l'any 1950, quan Alan Turing publica el llibre "\textit{Computing machinery and intelligence}". En aquest llibre es planteja, per primer cop, que les màquines poden pensar. Des d'aleshores, la IA ha acumulat successos i nefastos fracassos.\\
\begin{figure}[H]
    \centering
    \includegraphics[scale = 0.3]{images/Alan_Turing_Aged_16.jpg}
    \caption{Retrat d'Alan Turing}
    \label{fig:turing}
\end{figure}
S'entén per intel·ligència artificial la capacitat que té una màquina d'adquirir aptituds relacionades amb els éssers humans. Aquestes aptituds inclouen el raonament i l'aprenentatge, entre d'altres. Dit d'altra manera, la IA és un conjunt d'algorismes que pretén fer que les màquines imitin el comportament humà.\\
Segons els informàtics experts, \textit{Stuart Russell} (Figrua \ref{fig:russell}) i \textit{Peter Norvig} (Figura \ref{fig:norvig}), existeixen quatre tipus diferents d'intel·ligències artificials.
\begin{figure}[h!]
    \centering
    \begin{subfigure}[b]{0.48\linewidth}
        \includegraphics[width=\linewidth]{images/russell-aqua.jpg}
        \caption{Retrat de Stuart Russell}
        \label{fig:russell}
    \end{subfigure}
    \begin{subfigure}[b]{0.45\linewidth}
        \includegraphics[width=\linewidth]{images/peter_norvig_speaking_at_university_of_california_berkeley_2013.jpg}
        \caption{Retrat de Peter Norvig}
        \label{fig:norvig}
    \end{subfigure}
    \caption{Retrats dels científics Stuart Russell (esquerra) i Peter Norvig (dreta)}
    \label{fig:StuartNorvig}
\end{figure}
El primer tipus són aquells sistemes que pensen com els humans, és a dir, són aquells sistemes que automatitzen tasques com prendre decisions, resoldre problemes, etc. Un exemple d'aquest tipus de IA són les xarxes neuronals.\\
A continuació, es troben aquelles IA capaces de recrear el comportament humà. Aquest tipus de sistemes són computadores que duen a terme diferents tasques de la mateixa manera que ho faria un humà. El cas més evident són els robots.\\
En tercer lloc, es troben els sistemes de pensament racional, que són aquells que intenten copiar la lògica racional del pensament de les persones. Dit en altres paraules, es tracta d'aconseguir màquines que puguin disposar de la capacitat de percebre un estímul, jutjar-lo i actuar conseqüentment a aquell estímul. En aquest grup, s'engloben els sistemes experts. Aquests sistemes poden ser sistemes RBR (\textit{Rule Based Reasoning}), sistemes CBR (\textit{Case Based Reasoning}) o sistemes basats en xarxes bayesianes.\\
La quarta, i última forma en què es presenta la IA, són els sistemes que actuen de manera racional. Aquests sistemes són aquells que intenten copiar la forma de comportament humana. És el cas dels sistemes intel·ligents.
\subsection*{Avantatges i desavantatges de la IA}
\addcontentsline{toc}{subsection}{Avantatges i desavantatges de la IA}
Tenint en compte les defenses de la IA per part de \textit{Andy Chan} (Figure \ref{fig:chan}), \textit{Product Manager} de Infinia ML, i \textit{Kai-Fu Lee} (Figura \ref{fig:lee}), fundador del fons de capital de risc Sinovation Ventures, els avantatges de la intel·ligència artificial serien:
\begin{itemize}
    \item Automatització de processos.
    \item Potenciació de les tasques creatives.
    \item Precisió.
    \item Reducció de l'error humà.
    \item Reducció del temps emprat en l'anàlisi de dades.
    \item Manteniment predictiu.
    \item Millora en la presa de decisions a escala productiva i de negoci.
    \item Control i optimització de processos productius i línies de producció.
    \item Augment de la productivitat i de la qualitat de la producció.
\end{itemize}
Un cop vistos els avantatges, toca veure la part dolenta de la intel·ligència artificial. Com tot, no hi ha res perfecte i que no comporti riscos i conseqüències.
\begin{itemize}
    \item Disponibilitat de dades.
    \item Insuficiència de personal amb la qualificació adequada.
    \item Cost i temps d'implementació dels projectes.
\end{itemize}
\begin{figure}[h!]
    \centering
    \begin{subfigure}[b]{0.45\linewidth}
        \includegraphics[width=\linewidth]{images/1565701087600.jpg}
        \caption{Retrat d'Andy Chan}
        \label{fig:chan}
    \end{subfigure}
    \begin{subfigure}[b]{0.47\linewidth}
        \includegraphics[width=\linewidth]{images/Capture_medium.jpg}
        \caption{Retrat de Kai-Fu Lee}
        \label{fig:lee}
    \end{subfigure}
    \caption{Retrats d'Andy Chan (esquerra) i Kai-Fu Lee (dreta)}
    \label{fig:AndyKai}
\end{figure}
\subsection*{Qué és el \textit{Machine Learning}?}
\addcontentsline{toc}{subsection}{Que és el \textit{Machine Learning}?}
Es defineix com \textit{Machine Learning} la disciplina que atorga, als ordinadors, la capacitat d'aprendre de manera autònoma patrons en dades massives i elaborar prediccions.\\
En paraules de \textit{Jeff Hawkins} (Figura \ref{fig:hawkins}):
\begin{center}
    \begin{minipage}{0.9\linewidth}
        \vspace{5pt}
        {\small
            El \textit{Machine Learning} és la capacitat de predir el futur, per exemple, el pes d'un got que volem aixecar o la reacció de les persones als nostres actes, d'acord amb els patrons emmagatzemats en la memòria.
        }
        \vspace{5pt}
    \end{minipage}
\end{center}
\begin{figure}[H]
    \centering
    \includegraphics[scale = 0.3]{images/b2a43a485uokuohjp5l7peo9f0.jpg}
    \caption{Retrat de \textit{Jeff Hawkins}}
    \label{fig:hawkins}
\end{figure}
\subsubsection*{Tipus de \textit{Machine Learning}}
\addcontentsline{toc}{subsubsection}{Tipus de \textit{Machine Learning}}
Ara com ara, existeixen quatre tipus diferents d'aprenentatge automàtic.\\
El primer tipus és, l'aprenentatge supervisat. Aquest tipus d'aprenentatge, és aquell en el qual especifiquem a l'algorisme quina cosa ha d'aprendre i, un cop après, el model intel·ligent u seguirà al peu de la lletra. Dit més formalment, un model intel·ligent que utilitza aprenentatge supervisat és, aquell que ha estat entrenat amb una base de dades perfectament etiquetada i que realitza prediccions molt específiques basant-se en les etiquetes.\\
Dos exemples d'ús d'aquest tipus d'aprenentatge podrien ser els classificadors d'imatges i els classificadors de so.\\
El segon tipus d'aprenentatge és el no supervisat. L'objectiu d'aquest tipus d'aprenentatge és, que el programari aprengui per si mateix sense ajuda dels científics de dades a partir d'un conjunt de dades etiquetades. És a dir, aquesta tècnica de \textit{Machine Learning}, es basa en el fet que són els mateixos models els que troben patrons existents en les dades a analitzar.\\
Dins de l'aprenentatge no supervisat, es poden diferenciar dos subtipus. El primer és el \textit{Clustering} que permet identificar les agrupacions naturals resultants de l'anàlisi de totes les dades i, el segon, és l'\textit{Association} que permet descobrir les relacions entre les variables en un gran volum de dades.\\
El penúltim tipus de \textit{Machine Learning} és, l'aprenentatge semisupervisat. Aquest tipus d'aprenentatge és, una mescla dels dos anteriors, ja que utilitza un grup mínim d'etiquetes. La gran majoria de les dades són dades que no estan etiquetades, perquè tot i que les dades sense etiquetar augmenten els costos, són útils per aconseguir els objectius.\\
Tot i que sí que hi ha una supervisió per part del programador, no és una tasca que es realitzarà al llarg del procés d'aprenentatge (Algunes dades sí que s'etiqueten a mà, però la resta les etiquetarà l'algorisme).\\
Aquest tipus d'aprenentatge també es pot usar per a la classificació d'imatges. En aquest cas, es dona un subconjunt d'imatges etiquetades i un subconjunt sense etiquetes. El model s'ajudarà de les imatges etiquetades per classificar les no etiquetades.\\
Un segon exemple d'utilització d'aquesta tècnica, és la detecció de frau fiscal en transaccions financeres. El procediment és igual que abans; es tenen una sèrie de transaccions etiquetades com a frau (o com a no frau) i una sèrie de transaccions sense etiquetar. El model usarà les dades etiquetades per aprendre a detectar transaccions fraudulentes i també utilitzarà les dades no etiquetades per millorar la precisió a l'hora de detectar si una transacció és fraudulenta o no.\\
L'últim tipus d'aprenentatge automàtic és l'aprenentatge per reforç. La principal característica de l'aprenentatge per reforç és que és capaç de funcionar sense grans quantitats de dades. En aquesta tècnica, la IA guia el seu propi aprenentatge (explora un entorn desconegut) mitjançant un sistema de recompenses i càstigs. És a dir, és un sistema que guia el seu aprenentatge segons la tècnica de "prova i error".\\
El sistema de recompenses i càstigs comporta, no solament que, el model aprengui de manera immediata sinó que busqui maximitzar la recompensa.
Un exemple molt clar d'aquest tipus de sistemes són els robots. El robot rebrà una recompensa o un càstig segons realitzi bé, o no, una determinada acció. Un exemple de recompensa, per exemple, podria ser augmentar un punt en un sistema de puntuació. Per contra, el càstig, disminuiria un punt.\\
Aquest tipus d'aprenentatge també es pot extrapolar a la vida real. L'entrenament de gossos en seria un cas. L'entrenador ensenya una ordre a l'animal i aquest rebrà una galeta com a premi, si ho fa bé, o se li posarà el morrió si ho fa malament.
\subsection*{Qué és una xarxa neuronal?}
\addcontentsline{toc}{subsection}{Qué és una xarxa neuronal?}
Les xarxes neuronals són models matemàtics que imiten el processament d'una certa informació tal com ho faria el cervell humà. Atès que el seu objectiu és emular el comportament d'un cervell, una xarxa està composta per neurones i capes (igual que un cervell humà). Les neurones es transmeten senyals entre elles. Aquests senyals es transmeten des de l'entrada fins a la generació d'una sortida.\\
Segons la topologia de la xarxa, hi ha cinc possibles classificacions de xarxes neuronals artificials (ANN en anglès).\\
El primer tipus són les xarxes monocapa. Són les xarxes més senzilles, ja que només consten d'una capa d'entrada i una de sortida.\\
En segon lloc, hi ha les xarxes multicapa. Aquest tipus de xarxes són, una generalització de l'anterior. Aquestes contenen una sèrie de capes intermèdies entre la capa d'entrada i la de sortida. Aquestes capes intermèdies s'anomenen ocultes.\\
Depenent del nombre de connexions, la xarxa serà, o bé, completament connectada, o bé, parcialment connectada. Les xarxes completament connectades, són aquelles en les que cada neurona de cada capa està connectada a cada neurona de la següent capa (capes denses). En canvi, les xarxes parcialment connectades, són aquelles en les que no totes les neurones d'una capa estan connectades amb totes les neurones de la següent capa.\\
Un tercer tipus de xarxes neuronals, són les xarxes convolucionals (tècnica utilitzada per al desenvolupament dels diferents experiments). Aquest tipus de xarxes, són xarxes en les quals cada neurona no està connectada a totes les neurones de la següent capa. A més, les xarxes convolucionals (CNN en anglès), compten amb diverses capes ocultes especialitzades i amb una jerarquia. Això significa que les primeres capes detecten elements genèrics com poden ser línies o corbes i, tal com s'avança en les capes convolucionals, cada cop més especialitzades, s'arriba a les capes més profundes que reconeixen formes complexes com podria ser una cara.\\
El quart tipus de xarxes, són les xarxes recurrents. Aquestes xarxes no s'estructuren amb capes. Aquest tipus d'\textit{ANN} permeten connexions aleatòries que poden arribar a crear cicles, fet que permet que la xarxa tingui memòria.\\
Les dades es transformen i circulen per la xarxa no només en l'instant "t", sinó que també poden circular en l'instant "t + x", essent x un nombre natural.\\
L'últim tipus de xarxes neuronals són les xarxes de base radial. Aquest tipus de xarxes es caracteritza per tenir una capa d'entrada, una única capa oculta i una capa de sortida. Les neurones d'aquest tipus de xarxes usen les denominades funcions d'activació de base radial i la sortida de la xarxa es calcula com una combinació lineal de les sortides de cada neurona.
\subsubsection*{Qúe composa una xarxa neurona?}
\addcontentsline{toc}{subsubsection}{Qúe composa una xarxa neurona?}
La unitat de processament bàsica d'una xarxa neuronal són les neurones, que s'ordenen en capes.\\
Normalment, es poden diferenciar tres parts en una xarxa neuronal: la capa d'entrada, les capes ocultes i la capa de sortida.\\
La capa d'entrada, com el seu nom indica, representa tots els camps que s'entren a la xarxa per ser processats. Aquesta capa estarà formada per una o més neurones.\\
En segon lloc, estan les capes ocultes. Cal remarcar que aquest tipus de capes, poden no existir en la xarxa, és a dir, és possible que la xarxa estigui composta, exclusivament, per una capa d'entrada i una capa de sortida (xarxa monocapa). També existeix la possibilitat que, en comptes d'haver-hi diverses capes ocultes, solament n'hi hagi una. Cada capa oculta estarà formada per una o més neurones.\\
Finalment, la capa de sortida, és una capa que tindrà tantes neurones com resultats puguin tenir les dades processades. És a dir, en el cas d'aquest treball, la xarxa s'encarrega de classificar imatges entre quatre nivells diferents d'Alzheimer, per tant, la xarxa tindrà una totalitat de 4 neurones a la capa de sortida.\\
Com s'ha esmentat, les neurones estan organitzades per capes. Cada neurona es relaciona mitjançant "pesos" amb algunes (o totes) les neurones de la següent capa, així, les dades es presenten en la primera capa (la d'entrada) i els valors es propaguen des de la neurona en la qual es troben en aquell moment fins a totes aquelles neurones de la capa següent amb les quals hi hagi una relació. Finalment, els resultats arribaran a la capa de sortida, on es podran observar els resultats.
\subsubsection*{Com aprèn una xarxa neuronal?}
\addcontentsline{toc}{subsubsection}{Com aprèn una xarxa neuronal?}
Per entrenar una xarxa neuronal, cal exposar una gran quantitat d'entrades amb les seves respectives sortides. Durant l'entrenament, la xarxa anirà ajustant els pesos de les connexions entre neurones per tal de millorar el rendiment i poder prediccions en el futur. Per ajustar els pesos, la xarxa realitza complexos càlculs matemàtics que generen una sortida a partir d'una entrada per a posteriorment realitzar una comparació del càlcul obtingut amb la sortida real. Amb aquesta comparació de resultats s'obté un error que és l'utilitzat per ajustar els pesos de les connexions de la xarxa.\\
Aquest procés es repeteix un elevat nombre de vegades però no sempre amb la mateixa entrada i la mateixa sortida, sinó que les parelles \textit{input - output}, canvien. D'aquesta manera, al llarg del temps, la xarxa podrà ser capaç de realitzar, de manera precisa, prediccions amb dades d'entrada que no ha vist mai, és a dir, amb dades que no s'han utilitzat en el moment de l'entrenament.
\subsection*{Xarxes neuronals convolucionals (CNN)}
\addcontentsline{toc}{subsection}{Xarxes neuronals convolucionals (CNN)}

\subsection*{Entrenament de xarxes neuronals}
\addcontentsline{toc}{subsection}{Entrenament de xarxes neuronals}

\newpage

\section*{La intel·ligència artificial en la salut}
\addcontentsline{toc}{section}{La intel·ligència artificial en la salut}

\section*{Implementació de la proposta de treball i anàlisi de resultats}
\addcontentsline{toc}{section}{Implementació de la proposta de treball i anàlisi de resultats}

\subsection*{Disseny dels experiments}
\addcontentsline{toc}{subsection}{Disseny dels experiments}

\subsubsection*{Característiques de la màquina on s'han dut a terme els experiments}
\addcontentsline{toc}{subsubsection}{Característiques de la màquina on s'han dut a terme els experiments}

\subsubsection*{Llibreries de Python utilitzades}
\addcontentsline{toc}{subsubsection}{Llibreries de Python utilitzades}

\subsection*{Anàlisi de resultats}
\addcontentsline{toc}{subsection}{Anàlisi de resultats}

\subsubsection*{Graficació de la \textit{accuracy} i de la pèrdua}
\addcontentsline{toc}{subsubsection}{Graficació de la \textit{accuracy} i de la pèrdua}

\subsubsection*{Anàlisi de prediccions amb Lime}
\addcontentsline{toc}{subsubsection}{Anàlisi de prediccions amb Lime}

\section*{Conclusions}
\addcontentsline{toc}{section}{Conclusions}

\section*{Reptes i dificultats durant la realització del TFG}
\addcontentsline{toc}{section}{Reptes i dificultats durant la realització del TFG}

\section*{Annexs}
\addcontentsline{toc}{section}{Annexs}

\section*{Agraïments}
\addcontentsline{toc}{section}{Agraïments}
Jordi Planes (tutor i director del TFG), Nacho Lopez (Coordinador del GEI i encarregat d'acceptar el meu TFG), Joana Cervera (proporcionadora de informació a través d'xperiència personal), Esteban Cervera (proporcionador de informació a través d'xperiència personal), Sergi Cervera (conseller informàtic), Nuria Rosell (consellera d'estil), Nikita Deinega (conseller mèdic), 

\section*{Bibliografia i webgrafia}
\addcontentsline{toc}{section}{Bibliografia i webgrafia}
\begin{itemize}
    \item \href{https://www.kaggle.com/datasets/uraninjo/augmented-alzheimer-mri-dataset?resource=download}{\underline{Dataset}}
    \item \href{https://www.alz.org/alzheimer-demencia/que-es-la-enfermedad-de-alzheimer }{\underline{Que es la enfermedad de Alzheimer}}
    \item \href{https://www.caeme.org.ar/alzheimer-la-historia-de-una-enfermedad-que-desafia-a-la-ciencia/}{\underline{Alzheimer: la historia de una enfermedad que desafia a la ciencia }}
    \item \href{https://www.alz.org/alzheimer-demencia/tratamientos}{\underline{Alzheimer: tratamientos}}
    \item \href{https://www.mayoclinic.org/es-es/diseases-conditions/alzheimers-disease/in-depth/alzheimers/art-20048103}{\underline{Enfermedad de Alzheimer: los medicamentos ayudan a controlar los síntomas}}
    \item \href{http://www.alzfae.org/fundacion/549/medicamentos-autorizados}{\underline{Medicamentos autorizados}}
    \item \href{https://aiudo.es/asociaciones-de-alzheimer-y-centros/#asociaciones-de-alzheimer}{\underline{Asociaciones de Alzheimer y centros de ayuda en toda España}}
    \item \href{https://neurohouse.es/espacio-reimagine/la-inteligencia-artificial-una-nueva-aliada-contra-el-alzheimer}{\underline{La inteligencia artificial, una nueva aliada contra el Alzheimer}}
    \item \href{https://www.aecoc.es/innovation-hub-noticias/la-inteligencia-artificial-puede-detectar-signos-de-alzheimer-antes-que-nuestra-propia-familia/}{\underline{La inteligencia artificial puede detectar signos de Alzheimer antes que nuestra propia familia}}
    \item \href{https://www.lavanguardia.com/vida/20220427/8225520/inteligencia-artificial-ver-deterioro-cognitivo-acabara-alzheimer.html}{\underline{Inteligencia artficial para ver si deterioro cognitivo acabará en Alzheimer}}
    \item \href{https://www.larazon.es/sociedad/20220620/ajwe2fywxzgnta7zt4godk4bae.html}{\underline{Una nueva prueba para el diagnóstico precoz del alzhéimer logra una efectividad del 98\%}}
    \item \href{https://www.europarl.europa.eu/news/es/headlines/society/20200827STO85804/que-es-la-inteligencia-artificial-y-como-se-usa}{\underline{¿Qué es la inteligencia artificial y cómo se usa?}}
    \item \href{https://www.iberdrola.com/innovacion/que-es-inteligencia-artificial}{\underline{¿Qué es la inteligencia artificial?}}
    \item \href{https://nexusintegra.io/es/ventajas-y-desventajas-de-la-inteligencia-artificial/}{\underline{Ventajas y desventajas de la inteligencia artificial}}
    \item \href{https://www.santander.com/es/sala-de-comunicacion/dp/los-principales-retos-de-la-inteligencia-artificial}{\underline{Los principales retos de la inteligencia artificial}}
    \item \href{https://keepcoding.io/blog/tipos-de-aprendizaje-automatico/}{\underline{Tipos de aprendizaje automático}}
    \item \href{https://www.diegocalvo.es/clasificacion-de-redes-neuronales-artificiales/}{\underline{Clasificación de redes neuronales artificiales}}
    \item \href{https://keepcoding.io/blog/redes-neuronales-convolucionales/#Que_son_las_Redes_Neuronales_Convolucionales}{\underline{¿Qué són las redes neuronales convolucionales?}}
    \item \href{https://www.ibm.com/docs/es/spss-modeler/saas?topic=networks-neural-model}{\underline{El modelo de redes neuronales}}
    \item \href{https://www.juanbarrios.com/redes-neurales-convolucionales/ }{\underline{Redes neuronales convolucionales}}
    \item CEAFA, Censo de las personas con Alzheimer y otras demencias en España, Ministerio de derechos sociales y agenda 2030, 2019
    \item Mediagraphic, Enfermedad de Alzheimer. Clínica, diagnostico y neuropatologia, 2004, \href{https://www.medigraphic.com/newMedi/}{\underline{Mediagraphic - Literatura biomédic}}
    \item Alzheimer’s disease International, World Alzheimer report 2021. Journey through the diagnosis of dementia, 2021
    \item \href{https://www.alz.org/alzheimers-dementia/10_signs}{\underline{The 10 signs}}
    \item \href{https://www.cdc.gov/aging/spanish/features/dementia.html}{\underline{¿Qué es la demencia?}}
    \item \href{https://www.alzinfo.org/understand-alzheimers/clinical-stages-of-alzheimers/}{\underline{Clinical stages of Alzheimer's}}
    \item \href{https://www.ceafa.es/es/que-comunicamos/noticias/9-formas-de-reducir-el-riesgo-de-alzheimer}{\underline{9 formas de reducir el riesgo de Alzheimer}}
    \item INSALUD, Guía pràctica de la enfermedad de Alzheimer, 1996, Instituto nacional de la salud
    \item Alzheimer’s association, Información bàsica sobre la enfermedad de Alzheimer. Qué es y que se puede hacer, 2016
    \item \href{https://www.iberdrola.com/innovacion/machine-learning-aprendizaje-automatico}{\underline{Descubre los principales beneficios del "Machine Learning"}}
    \item \href{https://aws.amazon.com/es/what-is/neural-network/}{\underline{What is a neural network?}}
    \item \href{https://www.tensorflow.org/?hl=es-419}{\underline{TensorFlow}}
    \item \href{https://keras.io/}{\underline{Keras}}
    \item \href{https://www.w3schools.com/python/python_datetime.asp}{\underline{Datetime}}
    \item \href{https://docs.python.org/3/library/os.html}{\underline{OS}}
    \item \href{https://pypi.org/project/opencv-python/}{\underline{CV2}}
    \item \href{https://towardsdatascience.com/lime-how-to-interpret-machine-learning-models-with-python-94b0e7e4432e}{\underline{Lime}}
    \item \href{https://numpy.org/}{\underline{Numpy}}
    \item \href{https://matplotlib.org/}{\underline{Matplotlib}}
    \item \href{https://colab.research.google.com/}{\underline{Google Colab}}
    \item \href{https://upload.wikimedia.org/wikipedia/commons/a/a1/Alan_Turing_Aged_16.jpg}{\underline{Retrat d'Alan Turing}}
    \item \href{https://cs.berkeley.edu/sites/default/files/news_image/peter_norvig_speaking_at_university_of_california_berkeley_2013.jpg}{\underline{Retarat de \textit{Peter Norvig}}}
    \item \href{https://cs.berkeley.edu/sites/default/files/eecs_tout/russell-aqua.jpg}{\underline{Retrat de \textit{Stuart Rusell}}}
    \item \href{https://media.licdn.com/dms/image/C5603AQHgogiTxs2TZg/profile-displayphoto-shrink_800_800/0/1565701087600?e=2147483647&v=beta&t=p5ZjgearbT6mUGtGi2AEwDY07gaEWxmlGYDypO0HUVU}{\underline{Retrat d'\textit{Andy Chan}}}
    \item \href{https://upload.wikimedia.org/wikipedia/commons/0/0d/Capture_medium.jpg}{\underline{Retrat de \textit{Kai-Fu Lee}}}
    \item \href{https://m.media-amazon.com/images/W/IMAGERENDERING_521856-T1/images/S/amzn-author-media-prod/b2a43a485uokuohjp5l7peo9f0.jpg}{\underline{Retrat de \textit{Jeff Hawkins}}}
    \item \href{https://keepcoding.io/blog/redes-neuronales-convolucionales/#Que_son_las_Redes_Neuronales_Convolucionales}{\underline{Informació sobre les xarxes neuronals convolucionals}}
    \item \href{https://m.media-amazon.com/images/S/amzn-author-media-prod/b2a43a485uokuohjp5l7peo9f0._SX450_.jpg}{\underline{Retrat de \textit{Jeff Hawkins}}}
\end{itemize}


\end{document}
